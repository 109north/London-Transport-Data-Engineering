% Options for packages loaded elsewhere
\PassOptionsToPackage{unicode}{hyperref}
\PassOptionsToPackage{hyphens}{url}
%
\documentclass[
]{article}
\title{London TfL Accidents Report (2019)}
\usepackage{etoolbox}
\makeatletter
\providecommand{\subtitle}[1]{% add subtitle to \maketitle
  \apptocmd{\@title}{\par {\large #1 \par}}{}{}
}
\makeatother
\subtitle{LSE ME204 2023 - Final Project}
\author{Narayan Murti (202277950)}
\date{2023-07-31}

\usepackage{amsmath,amssymb}
\usepackage{lmodern}
\usepackage{iftex}
\ifPDFTeX
  \usepackage[T1]{fontenc}
  \usepackage[utf8]{inputenc}
  \usepackage{textcomp} % provide euro and other symbols
\else % if luatex or xetex
  \usepackage{unicode-math}
  \defaultfontfeatures{Scale=MatchLowercase}
  \defaultfontfeatures[\rmfamily]{Ligatures=TeX,Scale=1}
\fi
% Use upquote if available, for straight quotes in verbatim environments
\IfFileExists{upquote.sty}{\usepackage{upquote}}{}
\IfFileExists{microtype.sty}{% use microtype if available
  \usepackage[]{microtype}
  \UseMicrotypeSet[protrusion]{basicmath} % disable protrusion for tt fonts
}{}
\makeatletter
\@ifundefined{KOMAClassName}{% if non-KOMA class
  \IfFileExists{parskip.sty}{%
    \usepackage{parskip}
  }{% else
    \setlength{\parindent}{0pt}
    \setlength{\parskip}{6pt plus 2pt minus 1pt}}
}{% if KOMA class
  \KOMAoptions{parskip=half}}
\makeatother
\usepackage{xcolor}
\IfFileExists{xurl.sty}{\usepackage{xurl}}{} % add URL line breaks if available
\IfFileExists{bookmark.sty}{\usepackage{bookmark}}{\usepackage{hyperref}}
\hypersetup{
  pdftitle={London TfL Accidents Report (2019)},
  pdfauthor={Narayan Murti (202277950)},
  hidelinks,
  pdfcreator={LaTeX via pandoc}}
\urlstyle{same} % disable monospaced font for URLs
\usepackage[margin=1in]{geometry}
\usepackage{color}
\usepackage{fancyvrb}
\newcommand{\VerbBar}{|}
\newcommand{\VERB}{\Verb[commandchars=\\\{\}]}
\DefineVerbatimEnvironment{Highlighting}{Verbatim}{commandchars=\\\{\}}
% Add ',fontsize=\small' for more characters per line
\usepackage{framed}
\definecolor{shadecolor}{RGB}{248,248,248}
\newenvironment{Shaded}{\begin{snugshade}}{\end{snugshade}}
\newcommand{\AlertTok}[1]{\textcolor[rgb]{0.94,0.16,0.16}{#1}}
\newcommand{\AnnotationTok}[1]{\textcolor[rgb]{0.56,0.35,0.01}{\textbf{\textit{#1}}}}
\newcommand{\AttributeTok}[1]{\textcolor[rgb]{0.77,0.63,0.00}{#1}}
\newcommand{\BaseNTok}[1]{\textcolor[rgb]{0.00,0.00,0.81}{#1}}
\newcommand{\BuiltInTok}[1]{#1}
\newcommand{\CharTok}[1]{\textcolor[rgb]{0.31,0.60,0.02}{#1}}
\newcommand{\CommentTok}[1]{\textcolor[rgb]{0.56,0.35,0.01}{\textit{#1}}}
\newcommand{\CommentVarTok}[1]{\textcolor[rgb]{0.56,0.35,0.01}{\textbf{\textit{#1}}}}
\newcommand{\ConstantTok}[1]{\textcolor[rgb]{0.00,0.00,0.00}{#1}}
\newcommand{\ControlFlowTok}[1]{\textcolor[rgb]{0.13,0.29,0.53}{\textbf{#1}}}
\newcommand{\DataTypeTok}[1]{\textcolor[rgb]{0.13,0.29,0.53}{#1}}
\newcommand{\DecValTok}[1]{\textcolor[rgb]{0.00,0.00,0.81}{#1}}
\newcommand{\DocumentationTok}[1]{\textcolor[rgb]{0.56,0.35,0.01}{\textbf{\textit{#1}}}}
\newcommand{\ErrorTok}[1]{\textcolor[rgb]{0.64,0.00,0.00}{\textbf{#1}}}
\newcommand{\ExtensionTok}[1]{#1}
\newcommand{\FloatTok}[1]{\textcolor[rgb]{0.00,0.00,0.81}{#1}}
\newcommand{\FunctionTok}[1]{\textcolor[rgb]{0.00,0.00,0.00}{#1}}
\newcommand{\ImportTok}[1]{#1}
\newcommand{\InformationTok}[1]{\textcolor[rgb]{0.56,0.35,0.01}{\textbf{\textit{#1}}}}
\newcommand{\KeywordTok}[1]{\textcolor[rgb]{0.13,0.29,0.53}{\textbf{#1}}}
\newcommand{\NormalTok}[1]{#1}
\newcommand{\OperatorTok}[1]{\textcolor[rgb]{0.81,0.36,0.00}{\textbf{#1}}}
\newcommand{\OtherTok}[1]{\textcolor[rgb]{0.56,0.35,0.01}{#1}}
\newcommand{\PreprocessorTok}[1]{\textcolor[rgb]{0.56,0.35,0.01}{\textit{#1}}}
\newcommand{\RegionMarkerTok}[1]{#1}
\newcommand{\SpecialCharTok}[1]{\textcolor[rgb]{0.00,0.00,0.00}{#1}}
\newcommand{\SpecialStringTok}[1]{\textcolor[rgb]{0.31,0.60,0.02}{#1}}
\newcommand{\StringTok}[1]{\textcolor[rgb]{0.31,0.60,0.02}{#1}}
\newcommand{\VariableTok}[1]{\textcolor[rgb]{0.00,0.00,0.00}{#1}}
\newcommand{\VerbatimStringTok}[1]{\textcolor[rgb]{0.31,0.60,0.02}{#1}}
\newcommand{\WarningTok}[1]{\textcolor[rgb]{0.56,0.35,0.01}{\textbf{\textit{#1}}}}
\usepackage{graphicx}
\makeatletter
\def\maxwidth{\ifdim\Gin@nat@width>\linewidth\linewidth\else\Gin@nat@width\fi}
\def\maxheight{\ifdim\Gin@nat@height>\textheight\textheight\else\Gin@nat@height\fi}
\makeatother
% Scale images if necessary, so that they will not overflow the page
% margins by default, and it is still possible to overwrite the defaults
% using explicit options in \includegraphics[width, height, ...]{}
\setkeys{Gin}{width=\maxwidth,height=\maxheight,keepaspectratio}
% Set default figure placement to htbp
\makeatletter
\def\fps@figure{htbp}
\makeatother
\setlength{\emergencystretch}{3em} % prevent overfull lines
\providecommand{\tightlist}{%
  \setlength{\itemsep}{0pt}\setlength{\parskip}{0pt}}
\setcounter{secnumdepth}{-\maxdimen} % remove section numbering
\ifLuaTeX
  \usepackage{selnolig}  % disable illegal ligatures
\fi

\begin{document}
\maketitle

{
\setcounter{tocdepth}{2}
\tableofcontents
}
\hypertarget{setup}{%
\section{⚙️ Setup}\label{setup}}

I will discuss 2019 accident statistics from \emph{Transport for London}
(TfL), the governing body behind London's transport network.

\begin{Shaded}
\begin{Highlighting}[]
\NormalTok{knitr}\SpecialCharTok{::}\FunctionTok{include\_graphics}\NormalTok{(}\StringTok{"tfl.png"}\NormalTok{)}
\end{Highlighting}
\end{Shaded}

\includegraphics[width=5.56in]{tfl}

\hypertarget{import-libraries}{%
\paragraph{Import Libraries}\label{import-libraries}}

\begin{Shaded}
\begin{Highlighting}[]
\FunctionTok{library}\NormalTok{(httr2)}
\FunctionTok{library}\NormalTok{(jsonlite)}
\FunctionTok{library}\NormalTok{(rvest)}
\FunctionTok{library}\NormalTok{(tidyverse)}
\FunctionTok{library}\NormalTok{(DBI)}
\FunctionTok{library}\NormalTok{(RSQLite)}
\FunctionTok{library}\NormalTok{(dbplyr)}
\FunctionTok{library}\NormalTok{(lubridate)}
\FunctionTok{library}\NormalTok{(ggplot2)}
\FunctionTok{library}\NormalTok{(forcats)}
\end{Highlighting}
\end{Shaded}

\hypertarget{data-collection}{%
\paragraph{Data Collection}\label{data-collection}}

Save tfl accident stats into a json file in ``data/raw''

\begin{Shaded}
\begin{Highlighting}[]
\NormalTok{url }\OtherTok{\textless{}{-}} \StringTok{"https://api.tfl.gov.uk/AccidentStats/2019"}
  
\NormalTok{  response }\OtherTok{\textless{}{-}} \FunctionTok{request}\NormalTok{(url) }\SpecialCharTok{\%\textgreater{}\%} 
    \FunctionTok{req\_auth\_bearer\_token}\NormalTok{(}\StringTok{"c6a81dc38e4242dd8b16f65ee76d9206"}\NormalTok{) }\SpecialCharTok{\%\textgreater{}\%} 
    \FunctionTok{req\_perform}\NormalTok{()}
  
\NormalTok{  body }\OtherTok{\textless{}{-}}\NormalTok{ response }\SpecialCharTok{\%\textgreater{}\%} 
    \FunctionTok{resp\_body\_json}\NormalTok{()}
  
  \FunctionTok{write\_json}\NormalTok{(body, }\StringTok{"data/raw/tfl\_accident\_stats.json"}\NormalTok{)}
\end{Highlighting}
\end{Shaded}

Save borough population stats into a csv file in ``data/raw''

\begin{Shaded}
\begin{Highlighting}[]
\NormalTok{url }\OtherTok{\textless{}{-}} \StringTok{"http://www.citypopulation.de/en/uk/greaterlondon/"}
  
\NormalTok{  html }\OtherTok{\textless{}{-}} \FunctionTok{read\_html}\NormalTok{(url)}
  
\NormalTok{  borough\_table }\OtherTok{\textless{}{-}} \FunctionTok{html\_elements}\NormalTok{(html, }\StringTok{"table.data\#ts"}\NormalTok{) }\SpecialCharTok{\%\textgreater{}\%} 
    \FunctionTok{html\_table}\NormalTok{()}
  
  \FunctionTok{write\_csv}\NormalTok{(borough\_table[[}\DecValTok{1}\NormalTok{]], }\StringTok{"data/raw/borough\_population.csv"}\NormalTok{)}
\end{Highlighting}
\end{Shaded}

\hypertarget{preprocessing}{%
\paragraph{Preprocessing}\label{preprocessing}}

I need to make separate tables to complete my preprocessing because the
data under the ``casualties'' column is contained in a nested list. I
need a separate process to obtain this data, which I then will store
into the tfl.db database.

The following processes are required to preprocess these three tables in
the desired way:

\begin{quote}
\begin{itemize}
\tightlist
\item
  TfL API

  \begin{itemize}
  \tightlist
  \item
    Accidents Table

    \begin{itemize}
    \tightlist
    \item
      Create function to obtain data row-by-row from selected columns
    \item
      Save resulting df into the tfl.db database
    \end{itemize}
  \item
    Casualties Table

    \begin{itemize}
    \tightlist
    \item
      Create a similar function as above, but require a nested lapply
      function
    \item
      Save resulting df into the tfl.db database
    \end{itemize}
  \end{itemize}
\end{itemize}
\end{quote}

\begin{quote}
\begin{itemize}
\tightlist
\item
  Borough Population Website

  \begin{itemize}
  \tightlist
  \item
    Borough Population Table

    \begin{itemize}
    \tightlist
    \item
      Since the data was webscraped as an html table node, data is
      already in dataframe form
    \item
      Select desired columns and save resulting dataframe to tfl.db
      database
    \end{itemize}
  \end{itemize}
\end{itemize}
\end{quote}

Read in raw json file and csv file, then create a SQLite database
connection to store tables.

\begin{Shaded}
\begin{Highlighting}[]
\NormalTok{accident\_stats }\OtherTok{\textless{}{-}} \FunctionTok{read\_json}\NormalTok{(}\StringTok{"data/raw/tfl\_accident\_stats.json"}\NormalTok{)}
                 
\NormalTok{borough\_population }\OtherTok{\textless{}{-}} \FunctionTok{read\_csv}\NormalTok{(}\StringTok{"data/raw/borough\_population.csv"}\NormalTok{)}
\end{Highlighting}
\end{Shaded}

\begin{verbatim}
## New names:
## Rows: 34 Columns: 8
## -- Column specification
## -------------------------------------------------------- Delimiter: "," chr
## (3): Name, Status, ...8 num (5): PopulationEstimate1981-06-30,
## PopulationEstimate1991-06-30, Populat...
## i Use `spec()` to retrieve the full column specification for this data. i
## Specify the column types or set `show_col_types = FALSE` to quiet this message.
## * `` -> `...8`
\end{verbatim}

\begin{Shaded}
\begin{Highlighting}[]
\NormalTok{tfl\_db }\OtherTok{\textless{}{-}} \FunctionTok{dbConnect}\NormalTok{(}\AttributeTok{drv =}\NormalTok{ RSQLite}\SpecialCharTok{::}\FunctionTok{SQLite}\NormalTok{(),}
                 \StringTok{"data/tidy/tfl.db"}\NormalTok{) }
\end{Highlighting}
\end{Shaded}

Organize and store preferred tables into the SQLite database.

Accidents Table:

\begin{Shaded}
\begin{Highlighting}[]
\NormalTok{ selected\_cols }\OtherTok{\textless{}{-}} \FunctionTok{c}\NormalTok{(}\StringTok{"id"}\NormalTok{, }\StringTok{"lat"}\NormalTok{, }
                     \StringTok{"lon"}\NormalTok{, }\StringTok{"date"}\NormalTok{, }
                     \StringTok{"severity"}\NormalTok{, }\StringTok{"borough"}\NormalTok{)}
  
\NormalTok{  get\_row }\OtherTok{\textless{}{-}} \ControlFlowTok{function}\NormalTok{(list\_item) \{}
    
    \CommentTok{\# Select rows}
\NormalTok{    row }\OtherTok{\textless{}{-}}\NormalTok{ list\_item[selected\_cols]}
    
    \CommentTok{\# Treat any NULL cases}
\NormalTok{    null\_elements }\OtherTok{\textless{}{-}} \FunctionTok{sapply}\NormalTok{(row, is.null)}
\NormalTok{    row[null\_elements] }\OtherTok{\textless{}{-}} \ConstantTok{NA}
    
\NormalTok{    df }\OtherTok{\textless{}{-}} \FunctionTok{as\_tibble}\NormalTok{(row)}
    \FunctionTok{return}\NormalTok{(df)}
\NormalTok{  \}}
  
  \CommentTok{\# Create entire df}
\NormalTok{  df\_accidents }\OtherTok{\textless{}{-}} \FunctionTok{lapply}\NormalTok{(accident\_stats, get\_row) }\SpecialCharTok{\%\textgreater{}\%} 
    \FunctionTok{bind\_rows}\NormalTok{() }\SpecialCharTok{\%\textgreater{}\%} 
    \FunctionTok{unnest}\NormalTok{(}\AttributeTok{cols =} \FunctionTok{c}\NormalTok{(id, lat, lon, date, severity, borough))}
  
  \CommentTok{\# Change the \textasciigrave{}date\textasciigrave{} column into datetime format using lubridate}
\NormalTok{  df\_accidents }\OtherTok{\textless{}{-}}\NormalTok{ df\_accidents }\SpecialCharTok{\%\textgreater{}\%}
    \FunctionTok{mutate}\NormalTok{(}\AttributeTok{date =} \FunctionTok{ymd\_hms}\NormalTok{(date))}
  
  \CommentTok{\# Copy into tfl.db}
  \FunctionTok{copy\_to}\NormalTok{(tfl\_db, df\_accidents, }\StringTok{"accidents"}\NormalTok{,}
          \AttributeTok{types =} \FunctionTok{c}\NormalTok{(}\AttributeTok{id =} \StringTok{"int"}\NormalTok{,}
                    \AttributeTok{lat =} \StringTok{"double"}\NormalTok{,}
                    \AttributeTok{lon =} \StringTok{"double"}\NormalTok{,}
                    \AttributeTok{date =} \StringTok{"datetime"}\NormalTok{,}
                    \AttributeTok{severity =} \StringTok{"varchar(7)"}\NormalTok{,}
                    \AttributeTok{borough =} \StringTok{"varchar(22)"}\NormalTok{),}
          \AttributeTok{unique =} \FunctionTok{list}\NormalTok{(}\StringTok{"id"}\NormalTok{),}
          \AttributeTok{temporary =} \ConstantTok{FALSE}\NormalTok{, }\AttributeTok{overwrite =} \ConstantTok{TRUE}\NormalTok{)}
\end{Highlighting}
\end{Shaded}

Casualties Table:

\begin{Shaded}
\begin{Highlighting}[]
\NormalTok{ selected\_casualties }\OtherTok{\textless{}{-}} \FunctionTok{c}\NormalTok{(}\StringTok{"age"}\NormalTok{, }\StringTok{"class"}\NormalTok{, }
                           \StringTok{"severity"}\NormalTok{, }\StringTok{"mode"}\NormalTok{)}
  
  
\NormalTok{  get\_casualty }\OtherTok{\textless{}{-}} \ControlFlowTok{function}\NormalTok{(list\_item) \{}
    
    \CommentTok{\# Select the accidentId which I will need to add into the table}
\NormalTok{    accident\_id }\OtherTok{\textless{}{-}}\NormalTok{ list\_item[[}\StringTok{"id"}\NormalTok{]]}
    
    \CommentTok{\# Select casualties nested list}
\NormalTok{    casualties\_list }\OtherTok{\textless{}{-}}\NormalTok{ list\_item[[}\StringTok{"casualties"}\NormalTok{]]}
    
    \CommentTok{\# lapply to this list}
\NormalTok{    df }\OtherTok{\textless{}{-}} \FunctionTok{lapply}\NormalTok{(casualties\_list, }\ControlFlowTok{function}\NormalTok{(casualties\_list\_item) \{}
      
\NormalTok{      row }\OtherTok{\textless{}{-}}\NormalTok{ casualties\_list\_item[selected\_casualties]}
      
      \CommentTok{\# Treat any NULL cases}
\NormalTok{      null\_elements }\OtherTok{\textless{}{-}} \FunctionTok{sapply}\NormalTok{(row, is.null)}
\NormalTok{      row[null\_elements] }\OtherTok{\textless{}{-}} \ConstantTok{NA}
      
      \CommentTok{\# Ensure colnames are correct in case of an NA result}
\NormalTok{      row }\OtherTok{\textless{}{-}} \FunctionTok{setNames}\NormalTok{(row, selected\_casualties)}
      
      \CommentTok{\# Create each set of rows, being sure to add accidentId as well}
\NormalTok{      tibble }\OtherTok{\textless{}{-}} \FunctionTok{as\_tibble}\NormalTok{(row)}
\NormalTok{      tibble[}\StringTok{"accidentId"}\NormalTok{] }\OtherTok{\textless{}{-}}\NormalTok{ accident\_id}
      
      \FunctionTok{return}\NormalTok{(tibble)}
\NormalTok{    \}) }\SpecialCharTok{\%\textgreater{}\%} 
      \FunctionTok{bind\_rows}\NormalTok{()}
    
    \FunctionTok{return}\NormalTok{(df)}
\NormalTok{  \}}
  
  \CommentTok{\# Now bind tegether the entire df}
\NormalTok{  df\_casualties }\OtherTok{\textless{}{-}} \FunctionTok{lapply}\NormalTok{(accident\_stats, get\_casualty) }\SpecialCharTok{\%\textgreater{}\%}
    \FunctionTok{bind\_rows}\NormalTok{() }\SpecialCharTok{\%\textgreater{}\%}
    \FunctionTok{unnest}\NormalTok{(}\AttributeTok{cols =} \FunctionTok{c}\NormalTok{(age, class, severity, mode, accidentId)) }\SpecialCharTok{\%\textgreater{}\%}
    
    \CommentTok{\# Create the casualties id primary index}
    \FunctionTok{mutate}\NormalTok{(}\AttributeTok{id =} \FunctionTok{row\_number}\NormalTok{()) }\SpecialCharTok{\%\textgreater{}\%}
    \FunctionTok{select}\NormalTok{(id, accidentId, }\FunctionTok{everything}\NormalTok{()) }\SpecialCharTok{\%\textgreater{}\%}
    
    \CommentTok{\# Add spaces into the \textasciigrave{}mode\textasciigrave{} column to make more readable}
    \FunctionTok{mutate}\NormalTok{(}\AttributeTok{mode =} \FunctionTok{gsub}\NormalTok{(}\StringTok{"([a{-}z]?)([A{-}Z])"}\NormalTok{, }
                       \StringTok{"}\SpecialCharTok{\textbackslash{}\textbackslash{}}\StringTok{1 }\SpecialCharTok{\textbackslash{}\textbackslash{}}\StringTok{2"}\NormalTok{, }
\NormalTok{                       mode, }
                       \AttributeTok{perl =} \ConstantTok{TRUE}\NormalTok{)) }\SpecialCharTok{\%\textgreater{}\%}
    \FunctionTok{mutate}\NormalTok{(}\AttributeTok{mode =} \FunctionTok{sub}\NormalTok{(}\StringTok{" "}\NormalTok{, }\StringTok{""}\NormalTok{, mode))}
  
  \CommentTok{\# Copy into tfl.db}
  \FunctionTok{copy\_to}\NormalTok{(tfl\_db, df\_casualties, }\StringTok{"casualties"}\NormalTok{,}
          \AttributeTok{types =} \FunctionTok{c}\NormalTok{(}\AttributeTok{id =} \StringTok{"int"}\NormalTok{,}
                    \AttributeTok{accidentId =} \StringTok{"int"}\NormalTok{,}
                    \AttributeTok{age =} \StringTok{"tinyint"}\NormalTok{,}
                    \AttributeTok{class =} \StringTok{"varchar(10)"}\NormalTok{,}
                    \AttributeTok{severity =} \StringTok{"varchar(7)"}\NormalTok{,}
                    \AttributeTok{mode =} \StringTok{"varchar(20)"}\NormalTok{),}
          \AttributeTok{unique =} \FunctionTok{list}\NormalTok{(}\StringTok{"id"}\NormalTok{),}
          \AttributeTok{temporary =} \ConstantTok{FALSE}\NormalTok{, }\AttributeTok{overwrite =} \ConstantTok{TRUE}\NormalTok{)}
\end{Highlighting}
\end{Shaded}

Borough Population Table:

\begin{Shaded}
\begin{Highlighting}[]
 \CommentTok{\# Filter the desired columns and remove the Greater London row.}
\NormalTok{  borough\_population }\OtherTok{\textless{}{-}}\NormalTok{ borough\_population }\SpecialCharTok{\%\textgreater{}\%}
    \FunctionTok{select}\NormalTok{(}\StringTok{"borough"} \OtherTok{=} \StringTok{"Name"}\NormalTok{, }\StringTok{"population"} \OtherTok{=} \StringTok{"PopulationCensus2021{-}03{-}21"}\NormalTok{) }\SpecialCharTok{\%\textgreater{}\%}
    \FunctionTok{filter}\NormalTok{(borough }\SpecialCharTok{!=} \StringTok{"Greater London"}\NormalTok{)}
  
  \CommentTok{\# Now add this df into tfl.db}
  
  \FunctionTok{copy\_to}\NormalTok{(tfl\_db, borough\_population, }\StringTok{"borough\_population"}\NormalTok{,}
          \AttributeTok{types =} \FunctionTok{c}\NormalTok{(}\AttributeTok{borough =} \StringTok{"varchar(22)"}\NormalTok{,}
                    \AttributeTok{population =} \StringTok{"mediumint"}\NormalTok{),}
          \AttributeTok{unique =} \FunctionTok{list}\NormalTok{(}\StringTok{"borough"}\NormalTok{),}
          \AttributeTok{temporary =} \ConstantTok{FALSE}\NormalTok{, }\AttributeTok{overwrite =} \ConstantTok{TRUE}\NormalTok{)}
\end{Highlighting}
\end{Shaded}

\hypertarget{the-data}{%
\section{💾 The Data}\label{the-data}}

I accessed TfL's API to obtain 2019 accident statistics detailing
location, time, and casualty information (defined as people slightly,
severely, or fatally injured by the accident).

We can draw valuable conclusions from this data because by looking at
trends within different boroughs, different modes of transport, age
groups, and times of the year or day.

Here were the two tables I extracted from the TfL API:

\hypertarget{general-accident-information}{%
\paragraph{General Accident
Information}\label{general-accident-information}}

\begin{Shaded}
\begin{Highlighting}[]
\NormalTok{accidents\_tbl }\OtherTok{\textless{}{-}} \FunctionTok{tbl}\NormalTok{(tfl\_db, }\StringTok{"accidents"}\NormalTok{)}

\NormalTok{accidents\_tbl }\SpecialCharTok{\%\textgreater{}\%}
  \FunctionTok{head}\NormalTok{()}
\end{Highlighting}
\end{Shaded}

\begin{verbatim}
## # Source:   SQL [6 x 6]
## # Database: sqlite 3.41.2 [/Users/narayanmurti/Workspace/lse-me204-2023-final-project-narayanmurti/data/tidy/tfl.db]
##       id   lat     lon       date severity borough             
##    <int> <dbl>   <dbl>      <int> <chr>    <chr>               
## 1 345906  51.5 -0.0282 1546478400 Slight   Tower Hamlets       
## 2 345907  51.4 -0.118  1546469100 Slight   Croydon             
## 3 345908  51.5 -0.0727 1546454700 Slight   Tower Hamlets       
## 4 345909  51.5 -0.262  1546504860 Slight   Ealing              
## 5 345910  51.6 -0.136  1546504620 Slight   Islington           
## 6 345911  51.6  0.130  1546452600 Slight   Barking and Dagenham
\end{verbatim}

\hypertarget{casualty-stats}{%
\paragraph{Casualty Stats}\label{casualty-stats}}

\begin{Shaded}
\begin{Highlighting}[]
\NormalTok{casualties\_tbl }\OtherTok{\textless{}{-}} \FunctionTok{tbl}\NormalTok{(tfl\_db, }\StringTok{"casualties"}\NormalTok{)}

\NormalTok{casualties\_tbl }\SpecialCharTok{\%\textgreater{}\%}
  \FunctionTok{head}\NormalTok{()}
\end{Highlighting}
\end{Shaded}

\begin{verbatim}
## # Source:   SQL [6 x 6]
## # Database: sqlite 3.41.2 [/Users/narayanmurti/Workspace/lse-me204-2023-final-project-narayanmurti/data/tidy/tfl.db]
##      id accidentId   age class      severity mode               
##   <int>      <int> <int> <chr>      <chr>    <chr>              
## 1     1     345906    27 Driver     Slight   Car                
## 2     2     345907    42 Driver     Slight   Car                
## 3     3     345908    24 Driver     Slight   Pedal Cycle        
## 4     4     345909    48 Pedestrian Slight   Pedestrian         
## 5     5     345910    18 Driver     Slight   Powered Two Wheeler
## 6     6     345911    33 Driver     Slight   Car
\end{verbatim}

\hypertarget{borough-populations}{%
\paragraph{Borough Populations}\label{borough-populations}}

I then webscraped the HTML of a website containing census information
from each of London's boroughs. This enhanced conclusions we could draw
from the borough information in our accident statistics.

\begin{Shaded}
\begin{Highlighting}[]
\NormalTok{population\_tbl }\OtherTok{\textless{}{-}} \FunctionTok{tbl}\NormalTok{(tfl\_db, }\StringTok{"borough\_population"}\NormalTok{)}

\NormalTok{population\_tbl }\SpecialCharTok{\%\textgreater{}\%}
  \FunctionTok{head}\NormalTok{()}
\end{Highlighting}
\end{Shaded}

\begin{verbatim}
## # Source:   SQL [6 x 2]
## # Database: sqlite 3.41.2 [/Users/narayanmurti/Workspace/lse-me204-2023-final-project-narayanmurti/data/tidy/tfl.db]
##   borough              population
##   <chr>                     <int>
## 1 Barking and Dagenham     218869
## 2 Barnet                   389344
## 3 Bexley                   246472
## 4 Brent                    339816
## 5 Bromley                  329991
## 6 Camden                   210136
\end{verbatim}

\hypertarget{big-picture}{%
\section{🚁 Big Picture}\label{big-picture}}

\hypertarget{accidents-per-borough}{%
\paragraph{Accidents Per Borough}\label{accidents-per-borough}}

Accidents per borough gives us a general overview of which boroughs are
the biggest probelem areas for transportation accidents.

\textbf{Note: I omitted ``City of London'' from the bar chart on the
basis of it being an outlier due to abnormally small land area compared
to high traffic}

\begin{Shaded}
\begin{Highlighting}[]
\FunctionTok{ggplot}\NormalTok{(accidents\_tbl }\SpecialCharTok{\%\textgreater{}\%}
           
           \CommentTok{\# Note: for these two related graphs, we omit City of London on the }
           \CommentTok{\# basis of it being an outlier due to abnormally small land area }
           \CommentTok{\# compared to large foot traffic}
           \FunctionTok{filter}\NormalTok{(borough }\SpecialCharTok{!=} \StringTok{"City of London"}\NormalTok{), }
         \FunctionTok{aes}\NormalTok{(}\AttributeTok{y=}\FunctionTok{fct\_rev}\NormalTok{(}\FunctionTok{fct\_infreq}\NormalTok{(borough)))) }\SpecialCharTok{+}
    
    \FunctionTok{geom\_bar}\NormalTok{(}\FunctionTok{aes}\NormalTok{(}\AttributeTok{fill=}\NormalTok{severity)) }\SpecialCharTok{+}
    
    \FunctionTok{scale\_fill\_manual}\NormalTok{(}\AttributeTok{values =} \FunctionTok{c}\NormalTok{(}\StringTok{"Slight"} \OtherTok{=} \StringTok{"\#2E86C1"}\NormalTok{, }
                                 \StringTok{"Serious"} \OtherTok{=} \StringTok{"\#F39C12"}\NormalTok{, }
                                 \StringTok{"Fatal"} \OtherTok{=} \StringTok{"\#B03A2E"}\NormalTok{),}
                      \AttributeTok{name =} \StringTok{"Severity"}\NormalTok{) }\SpecialCharTok{+}
    
    \FunctionTok{scale\_x\_continuous}\NormalTok{(}\AttributeTok{breaks =} \FunctionTok{seq}\NormalTok{(}\DecValTok{0}\NormalTok{, }\DecValTok{3000}\NormalTok{, }\DecValTok{1000}\NormalTok{),}
                       \AttributeTok{labels =}\NormalTok{ scales}\SpecialCharTok{::}\FunctionTok{unit\_format}\NormalTok{(}\AttributeTok{unit=}\StringTok{"K"}\NormalTok{, }\AttributeTok{scale =} \FloatTok{1e{-}3}\NormalTok{)) }\SpecialCharTok{+}
    
    \FunctionTok{labs}\NormalTok{(}\AttributeTok{title =} \StringTok{"Gross London Transportation Accidents in 2019"}\NormalTok{,}
         \AttributeTok{subtitle =} \StringTok{"City of Westminster leads the pack by over 1,000 cases."}\NormalTok{,}
         \AttributeTok{x =} \StringTok{"Number of accidents"}\NormalTok{,}
         \AttributeTok{y =} \StringTok{"Borough"}\NormalTok{) }\SpecialCharTok{+}
    
    \FunctionTok{theme\_minimal}\NormalTok{() }\SpecialCharTok{+}
    \FunctionTok{theme}\NormalTok{(}\AttributeTok{axis.title =} \FunctionTok{element\_text}\NormalTok{(}\AttributeTok{size =} \FunctionTok{rel}\NormalTok{(}\FloatTok{1.2}\NormalTok{)),}
          \AttributeTok{axis.title.x =} \FunctionTok{element\_text}\NormalTok{(}\AttributeTok{margin =} \FunctionTok{margin}\NormalTok{(}\AttributeTok{t=}\DecValTok{10}\NormalTok{)),}
          \AttributeTok{axis.title.y =} \FunctionTok{element\_text}\NormalTok{(}\AttributeTok{margin =} \FunctionTok{margin}\NormalTok{(}\AttributeTok{t=}\DecValTok{0}\NormalTok{)),}
          \AttributeTok{plot.title =} \FunctionTok{element\_text}\NormalTok{(}\AttributeTok{size=}\FunctionTok{rel}\NormalTok{(}\FloatTok{1.6}\NormalTok{), }\AttributeTok{color=}\StringTok{"\#5B2C6F"}\NormalTok{))}
\end{Highlighting}
\end{Shaded}

\includegraphics{publishreport_files/figure-latex/unnamed-chunk-12-1.pdf}

\hypertarget{per-capita-accidents-per-borough}{%
\paragraph{Per Capita Accidents Per
Borough}\label{per-capita-accidents-per-borough}}

To get a better understanding of the significance of the above plot, we
should also examine accidents per capita using the borough population
data we webscraped seperately from the TfL API. To setup this chart, I
performed an inner join between the Borough Population Table and
Accidents Table on the ``borough'' columns.

I then computed accidents per capita for each borough and added that as
a column in this new dataframe.

Our resulting plot shows an interesting result that City of Westminster
still leads the rest of the boroughs by a considerable amount, even when
borough population is taken into account. We will explore more
specifically about City of Westminster in the following section.

\textbf{Note: Observe same ``City of London'' note as above}

\begin{Shaded}
\begin{Highlighting}[]
\NormalTok{plot\_df }\OtherTok{\textless{}{-}}\NormalTok{ accidents\_tbl }\SpecialCharTok{\%\textgreater{}\%} 
    \FunctionTok{group\_by}\NormalTok{(borough) }\SpecialCharTok{\%\textgreater{}\%} 
    \FunctionTok{count}\NormalTok{() }\SpecialCharTok{\%\textgreater{}\%}
    \FunctionTok{inner\_join}\NormalTok{(population\_tbl) }\SpecialCharTok{\%\textgreater{}\%}
    \FunctionTok{mutate}\NormalTok{(}\StringTok{"accidents\_per\_capita"} \OtherTok{=}\NormalTok{ (}\FunctionTok{as.double}\NormalTok{(n)}\SpecialCharTok{/}\FunctionTok{as.double}\NormalTok{(population))) }\SpecialCharTok{\%\textgreater{}\%}
    \FunctionTok{select}\NormalTok{(borough, accidents\_per\_capita) }\SpecialCharTok{\%\textgreater{}\%}
    
    \CommentTok{\# Note: removed City of London borough row because its low population \& size }
    \CommentTok{\# relative to its high foot traffic skewed this graph considerably.}
    \FunctionTok{filter}\NormalTok{(borough }\SpecialCharTok{!=} \StringTok{"City of London"}\NormalTok{)}
\end{Highlighting}
\end{Shaded}

\begin{verbatim}
## Joining with `by = join_by(borough)`
\end{verbatim}

\begin{Shaded}
\begin{Highlighting}[]
  \FunctionTok{ggplot}\NormalTok{(plot\_df, }\FunctionTok{aes}\NormalTok{(}\AttributeTok{y=}\FunctionTok{fct\_reorder}\NormalTok{(borough, accidents\_per\_capita), }
                      \AttributeTok{x=}\NormalTok{accidents\_per\_capita)) }\SpecialCharTok{+}
    
    \FunctionTok{geom\_col}\NormalTok{(}\FunctionTok{aes}\NormalTok{(}\AttributeTok{fill =}\NormalTok{ accidents\_per\_capita)) }\SpecialCharTok{+}
    
    \FunctionTok{scale\_fill\_gradient}\NormalTok{(}\AttributeTok{low =} \StringTok{"\#FDEDEC"}\NormalTok{,}
                        \AttributeTok{high =} \StringTok{"\#CB4335"}\NormalTok{) }\SpecialCharTok{+}
    
    \FunctionTok{labs}\NormalTok{(}\AttributeTok{title =} \StringTok{"Per Capita London Transportation Accidents in 2019"}\NormalTok{,}
         \AttributeTok{subtitle =} \StringTok{"City of Westminister also leads the pack in accidents per capita."}\NormalTok{,}
         \AttributeTok{x =} \StringTok{"Accidents Per Capita"}\NormalTok{,}
         \AttributeTok{y =} \StringTok{"Borough"}\NormalTok{) }\SpecialCharTok{+}
    
    \FunctionTok{guides}\NormalTok{(}\AttributeTok{fill =} \ConstantTok{FALSE}\NormalTok{) }\SpecialCharTok{+}
    
    \FunctionTok{theme\_minimal}\NormalTok{() }\SpecialCharTok{+}
    \FunctionTok{theme}\NormalTok{(}\AttributeTok{axis.title =} \FunctionTok{element\_text}\NormalTok{(}\AttributeTok{size =} \FunctionTok{rel}\NormalTok{(}\FloatTok{1.2}\NormalTok{)),}
          \AttributeTok{axis.title.x =} \FunctionTok{element\_text}\NormalTok{(}\AttributeTok{margin =} \FunctionTok{margin}\NormalTok{(}\AttributeTok{t=}\DecValTok{10}\NormalTok{)),}
          \AttributeTok{axis.title.y =} \FunctionTok{element\_text}\NormalTok{(}\AttributeTok{margin =} \FunctionTok{margin}\NormalTok{(}\AttributeTok{t=}\DecValTok{0}\NormalTok{)),}
          \AttributeTok{plot.title =} \FunctionTok{element\_text}\NormalTok{(}\AttributeTok{size=}\FunctionTok{rel}\NormalTok{(}\FloatTok{1.6}\NormalTok{), }\AttributeTok{color=}\StringTok{"\#5B2C6F"}\NormalTok{))}
\end{Highlighting}
\end{Shaded}

\begin{verbatim}
## Warning: The `<scale>` argument of `guides()` cannot be `FALSE`. Use "none" instead as
## of ggplot2 3.3.4.
## This warning is displayed once every 8 hours.
## Call `lifecycle::last_lifecycle_warnings()` to see where this warning was
## generated.
\end{verbatim}

\includegraphics{publishreport_files/figure-latex/unnamed-chunk-13-1.pdf}

\hypertarget{accidents-throughout-the-year}{%
\paragraph{Accidents Throughout the
Year}\label{accidents-throughout-the-year}}

To get an overview of the time data, we will examine number of accidents
throughout the year. In order to do this, I needed to convert my date
column into datetime format, then omit the time information from this
column with the lubridate package.

This is a useful illustration that shows how the number of accidents
fluctuated throughout 2019. It shows that accidents peak around the end
of the year and fall lowest at the beginning of the year.

\begin{Shaded}
\begin{Highlighting}[]
\NormalTok{plot\_df }\OtherTok{\textless{}{-}}\NormalTok{ accidents\_tbl }\SpecialCharTok{\%\textgreater{}\%} 
    \FunctionTok{collect}\NormalTok{() }\SpecialCharTok{\%\textgreater{}\%} 
    \FunctionTok{mutate}\NormalTok{(}\AttributeTok{date =} \FunctionTok{as\_datetime}\NormalTok{(date)) }\SpecialCharTok{\%\textgreater{}\%}
    \FunctionTok{mutate}\NormalTok{(}\AttributeTok{date =} \FunctionTok{date}\NormalTok{(date)) }\SpecialCharTok{\%\textgreater{}\%}
    \FunctionTok{group\_by}\NormalTok{(date) }\SpecialCharTok{\%\textgreater{}\%}
    \FunctionTok{count}\NormalTok{()}
  
  \FunctionTok{ggplot}\NormalTok{(plot\_df, }\FunctionTok{aes}\NormalTok{(}\AttributeTok{x=}\NormalTok{date, }\AttributeTok{y=}\NormalTok{n)) }\SpecialCharTok{+}
    
    \FunctionTok{geom\_col}\NormalTok{(}\AttributeTok{fill =} \StringTok{"\#2E86C1"}\NormalTok{) }\SpecialCharTok{+}
    
    \FunctionTok{geom\_smooth}\NormalTok{(}\AttributeTok{method =} \StringTok{"gam"}\NormalTok{, }\AttributeTok{level =} \FloatTok{0.95}\NormalTok{, }\AttributeTok{color =} \StringTok{"\#A569BD"}\NormalTok{) }\SpecialCharTok{+}
    
    \FunctionTok{scale\_x\_date}\NormalTok{(}\AttributeTok{date\_breaks=}\StringTok{"1 month"}\NormalTok{, }\AttributeTok{date\_labels=}\StringTok{"\%B"}\NormalTok{) }\SpecialCharTok{+}
    
    \FunctionTok{labs}\NormalTok{(}\AttributeTok{title =} \StringTok{"November Saw the Most Accidents Per Month in 2019"}\NormalTok{,}
         \AttributeTok{x =} \StringTok{"Month (2019)"}\NormalTok{,}
         \AttributeTok{y =} \StringTok{"Accidents"}\NormalTok{) }\SpecialCharTok{+}
    
    \FunctionTok{theme\_minimal}\NormalTok{() }\SpecialCharTok{+}
    \FunctionTok{theme}\NormalTok{(}\AttributeTok{axis.text.x =} \FunctionTok{element\_text}\NormalTok{(}\AttributeTok{size =} \FunctionTok{rel}\NormalTok{(}\FloatTok{1.2}\NormalTok{)),}
          \AttributeTok{plot.title =} \FunctionTok{element\_text}\NormalTok{(}\AttributeTok{size=}\FunctionTok{rel}\NormalTok{(}\FloatTok{1.6}\NormalTok{), }\AttributeTok{color=}\StringTok{"\#5B2C6F"}\NormalTok{))}
\end{Highlighting}
\end{Shaded}

\includegraphics{publishreport_files/figure-latex/unnamed-chunk-14-1.pdf}

\hypertarget{casualties-by-mode-of-transport}{%
\paragraph{Casualties by Mode of
Transport}\label{casualties-by-mode-of-transport}}

We now want to find out how these injuries are happening, so we will
inspect the mode column of the Casualties Table. This graph shows us the
mode by which the injured person was traveling. So, ``Car'' means that
the injured person was a driver or passenger in a car, ``Pedestrian''
means that the injured person was traveling on foot, etc.

We see that drivers or passengers of cars experienced the most amount of
casualties over other methods of transportation. It is important to
remember that TfL defines ``casualties'' not as deaths, but as injuries
that are either slight, serious, or fatal.

\begin{Shaded}
\begin{Highlighting}[]
\FunctionTok{ggplot}\NormalTok{(casualties\_tbl, }\FunctionTok{aes}\NormalTok{(}\AttributeTok{y=}\FunctionTok{fct\_rev}\NormalTok{(}\FunctionTok{fct\_infreq}\NormalTok{(mode)))) }\SpecialCharTok{+}
    
    \FunctionTok{geom\_bar}\NormalTok{(}\FunctionTok{aes}\NormalTok{(}\AttributeTok{fill=}\NormalTok{severity)) }\SpecialCharTok{+}
    
    \FunctionTok{scale\_fill\_manual}\NormalTok{(}\AttributeTok{values =} \FunctionTok{c}\NormalTok{(}\StringTok{"Slight"} \OtherTok{=} \StringTok{"\#2E86C1"}\NormalTok{, }
                                 \StringTok{"Serious"} \OtherTok{=} \StringTok{"\#F39C12"}\NormalTok{, }
                                 \StringTok{"Fatal"} \OtherTok{=} \StringTok{"\#B03A2E"}\NormalTok{),}
                      \AttributeTok{name =} \StringTok{"Severity"}\NormalTok{) }\SpecialCharTok{+}
    
    \FunctionTok{scale\_x\_continuous}\NormalTok{(}\AttributeTok{breaks =} \FunctionTok{seq}\NormalTok{(}\DecValTok{5000}\NormalTok{, }\DecValTok{25000}\NormalTok{, }\DecValTok{5000}\NormalTok{),}
                       \AttributeTok{labels =}\NormalTok{ scales}\SpecialCharTok{::}\FunctionTok{unit\_format}\NormalTok{(}\AttributeTok{unit=}\StringTok{"K"}\NormalTok{, }\AttributeTok{scale =} \FloatTok{1e{-}3}\NormalTok{)) }\SpecialCharTok{+}
    
    \FunctionTok{labs}\NormalTok{(}\AttributeTok{title =} \StringTok{"Cars are the leading mode by which casualties (injuries) occur on TfL systems"}\NormalTok{,}
         \AttributeTok{x =} \StringTok{"Casualties"}\NormalTok{,}
         \AttributeTok{y =} \StringTok{"Mode of Transport"}\NormalTok{) }\SpecialCharTok{+}
    
    \FunctionTok{theme\_minimal}\NormalTok{() }\SpecialCharTok{+} 
    \FunctionTok{theme}\NormalTok{(}\AttributeTok{axis.text.x =} \FunctionTok{element\_text}\NormalTok{(}\AttributeTok{size =} \FunctionTok{rel}\NormalTok{(}\FloatTok{1.2}\NormalTok{)),}
          \AttributeTok{plot.title =} \FunctionTok{element\_text}\NormalTok{(}\AttributeTok{size=}\FunctionTok{rel}\NormalTok{(}\FloatTok{1.6}\NormalTok{), }\AttributeTok{color=}\StringTok{"\#5B2C6F"}\NormalTok{))}
\end{Highlighting}
\end{Shaded}

\includegraphics{publishreport_files/figure-latex/unnamed-chunk-15-1.pdf}

\hypertarget{serious-or-fatal-casualties-by-mode-of-transport}{%
\paragraph{Serious or Fatal Casualties by Mode of
Transport}\label{serious-or-fatal-casualties-by-mode-of-transport}}

When we restrict the Casualties Table to only include serious or fatal
casualties, we see an interesting result: Pedestrian, Motorcycle, and
Bicycle casualties overtake Car casualties when ``slight'' casualties
are not considered. This may mean that Car injuries are more frequent,
yet Pedestrian, motorcycle, and bicycle injuries are more dangerous.

\begin{Shaded}
\begin{Highlighting}[]
\FunctionTok{ggplot}\NormalTok{(casualties\_tbl }\SpecialCharTok{\%\textgreater{}\%}
           \FunctionTok{filter}\NormalTok{(severity }\SpecialCharTok{!=} \StringTok{"Slight"}\NormalTok{), }
         \FunctionTok{aes}\NormalTok{(}\AttributeTok{y=}\FunctionTok{fct\_rev}\NormalTok{(}\FunctionTok{fct\_infreq}\NormalTok{(mode)))) }\SpecialCharTok{+}
    
    \FunctionTok{geom\_bar}\NormalTok{(}\FunctionTok{aes}\NormalTok{(}\AttributeTok{fill=}\NormalTok{severity)) }\SpecialCharTok{+}
    
    \FunctionTok{scale\_fill\_manual}\NormalTok{(}\AttributeTok{values =} \FunctionTok{c}\NormalTok{(}\StringTok{"Serious"} \OtherTok{=} \StringTok{"\#F39C12"}\NormalTok{, }
                                 \StringTok{"Fatal"} \OtherTok{=} \StringTok{"\#B03A2E"}\NormalTok{),}
                      \AttributeTok{name =} \StringTok{"Severity"}\NormalTok{) }\SpecialCharTok{+}
    
    \FunctionTok{scale\_x\_continuous}\NormalTok{(}\AttributeTok{breaks =} \FunctionTok{seq}\NormalTok{(}\DecValTok{1000}\NormalTok{, }\DecValTok{3000}\NormalTok{, }\DecValTok{1000}\NormalTok{),}
                       \AttributeTok{labels =}\NormalTok{ scales}\SpecialCharTok{::}\FunctionTok{unit\_format}\NormalTok{(}\AttributeTok{unit=}\StringTok{"K"}\NormalTok{, }\AttributeTok{scale =} \FloatTok{1e{-}3}\NormalTok{)) }\SpecialCharTok{+}
    
    \FunctionTok{labs}\NormalTok{(}\AttributeTok{title =} \StringTok{"Pedestrians experience the most amount of serious or fatal injuries over other modes of transport"}\NormalTok{,}
         \AttributeTok{x =} \StringTok{"Casualties (injuries)"}\NormalTok{,}
         \AttributeTok{y =} \StringTok{"Mode of Transport"}\NormalTok{) }\SpecialCharTok{+}
    
    \FunctionTok{theme\_minimal}\NormalTok{() }\SpecialCharTok{+}
    \FunctionTok{theme}\NormalTok{(}\AttributeTok{axis.text.x =} \FunctionTok{element\_text}\NormalTok{(}\AttributeTok{size =} \FunctionTok{rel}\NormalTok{(}\FloatTok{1.2}\NormalTok{)),}
          \AttributeTok{plot.title =} \FunctionTok{element\_text}\NormalTok{(}\AttributeTok{size=}\FunctionTok{rel}\NormalTok{(}\FloatTok{1.6}\NormalTok{), }\AttributeTok{color=}\StringTok{"\#5B2C6F"}\NormalTok{))}
\end{Highlighting}
\end{Shaded}

\includegraphics{publishreport_files/figure-latex/unnamed-chunk-16-1.pdf}

\hypertarget{further-exploratory-analysis}{%
\section{📈 Further Exploratory
Analysis}\label{further-exploratory-analysis}}

\hypertarget{accidents-per-hour}{%
\paragraph{Accidents Per Hour}\label{accidents-per-hour}}

We want to find out what times of day led to the most accidents in 2019.
To create this ciruclar barplot, I needed to apply the
\texttt{coord\_polar()} function to the barplot of my preprocessed
dataframe.

Using lubridate, I converted my date column to datetime format, then
pulled only the hour element from my date data into a new hour column.
Then, grouping by hour, I was albe to visualize the problematic hours.
The morning and evening spikes can likely be attributed to rush hour
transport, which is what we expect.

It is interesting that the 8am spike is much more abrupt than the
gradual 3-6pm spike. This shows how it gradually gets busier in the
evenings, followed by a sharp drop off after 7pm.

\begin{Shaded}
\begin{Highlighting}[]
\NormalTok{plot\_df }\OtherTok{\textless{}{-}}\NormalTok{ accidents\_tbl }\SpecialCharTok{\%\textgreater{}\%} 
    \FunctionTok{collect}\NormalTok{() }\SpecialCharTok{\%\textgreater{}\%} 
    \FunctionTok{mutate}\NormalTok{(}\AttributeTok{date =} \FunctionTok{as\_datetime}\NormalTok{(date)) }\SpecialCharTok{\%\textgreater{}\%}
    \FunctionTok{mutate}\NormalTok{(}\AttributeTok{hour =} \FunctionTok{hour}\NormalTok{(date)) }\SpecialCharTok{\%\textgreater{}\%}
    \FunctionTok{group\_by}\NormalTok{(hour)}
  
  
  \FunctionTok{ggplot}\NormalTok{(plot\_df, }\FunctionTok{aes}\NormalTok{(}\AttributeTok{x=}\NormalTok{hour)) }\SpecialCharTok{+}
    
    \FunctionTok{geom\_bar}\NormalTok{(}\FunctionTok{aes}\NormalTok{(}\AttributeTok{fill =}\NormalTok{ severity)) }\SpecialCharTok{+}
    
    \FunctionTok{annotate}\NormalTok{(}\StringTok{\textquotesingle{}segment\textquotesingle{}}\NormalTok{, }
             \AttributeTok{x=} \FunctionTok{seq}\NormalTok{(}\DecValTok{0}\NormalTok{, }\DecValTok{23}\NormalTok{),}
             \AttributeTok{y =} \DecValTok{0}\NormalTok{, }
             \AttributeTok{xend =} \FunctionTok{seq}\NormalTok{(}\DecValTok{0}\NormalTok{, }\DecValTok{23}\NormalTok{),}
             \AttributeTok{yend =} \DecValTok{4204}\NormalTok{, }
             
             \AttributeTok{alpha =} \FloatTok{0.2}\NormalTok{) }\SpecialCharTok{+}

    \FunctionTok{scale\_y\_continuous}\NormalTok{(}
      \AttributeTok{limits =} \FunctionTok{c}\NormalTok{(}\SpecialCharTok{{-}}\DecValTok{1000}\NormalTok{, }\DecValTok{4204}\NormalTok{),}
      \AttributeTok{expand =} \FunctionTok{c}\NormalTok{(}\DecValTok{0}\NormalTok{, }\DecValTok{0}\NormalTok{)) }\SpecialCharTok{+}
    
    \FunctionTok{scale\_x\_continuous}\NormalTok{(}\AttributeTok{breaks =} \FunctionTok{seq}\NormalTok{(}\DecValTok{0}\NormalTok{, }\DecValTok{23}\NormalTok{)) }\SpecialCharTok{+}
    
    \FunctionTok{labs}\NormalTok{(}\AttributeTok{title =} \StringTok{"Most Accidents Occer Between 3{-}6pm"}\NormalTok{,}
         \AttributeTok{subtitle =} \StringTok{"Accidents also spike at 8am"}\NormalTok{,}
         \AttributeTok{x =} \StringTok{"Hour"}\NormalTok{,}
         \AttributeTok{y =} \StringTok{""}\NormalTok{) }\SpecialCharTok{+}
    
    \FunctionTok{theme\_minimal}\NormalTok{() }\SpecialCharTok{+}
    \FunctionTok{theme}\NormalTok{(}\AttributeTok{plot.title =} \FunctionTok{element\_text}\NormalTok{(}\AttributeTok{size =} \FunctionTok{rel}\NormalTok{(}\FloatTok{1.8}\NormalTok{),}
                                    \AttributeTok{color=}\StringTok{"\#5B2C6F"}\NormalTok{),}
          \AttributeTok{axis.text.x =} \FunctionTok{element\_text}\NormalTok{(}\AttributeTok{size =} \FunctionTok{rel}\NormalTok{(}\FloatTok{1.5}\NormalTok{)),}
          \AttributeTok{axis.text.y =} \FunctionTok{element\_blank}\NormalTok{()) }\SpecialCharTok{+}
    
    \FunctionTok{coord\_polar}\NormalTok{() }
\end{Highlighting}
\end{Shaded}

\includegraphics{publishreport_files/figure-latex/unnamed-chunk-17-1.pdf}

The following four plots use an inner join on \texttt{accidentsId}
between the Accidents and Casualties tables to display information about
the age ranges of those injured.

They are all followed by different preprocessing procedures that extract
the specific information we want for each given graph.

Also, each of the following plots contained a mutate call including the
\texttt{cut()} function to sort the ages in the data into 10-year bins
so that the visualization was more informative.

\hypertarget{age-range-for-all-boroughs}{%
\paragraph{Age Range for All
Boroughs}\label{age-range-for-all-boroughs}}

We see that 21-30 year olds experience the highest number of transport
accidents across all boroughs.

\begin{Shaded}
\begin{Highlighting}[]
\NormalTok{plot\_df }\OtherTok{\textless{}{-}}\NormalTok{ accidents\_tbl }\SpecialCharTok{\%\textgreater{}\%}
    
    \FunctionTok{inner\_join}\NormalTok{(casualties\_tbl, }
               \AttributeTok{by =} \FunctionTok{c}\NormalTok{(}\StringTok{"id"} \OtherTok{=} \StringTok{"accidentId"}\NormalTok{)) }\SpecialCharTok{\%\textgreater{}\%}
    
    \FunctionTok{select}\NormalTok{(borough, }
\NormalTok{           age, }
           \StringTok{"severity"} \OtherTok{=}\NormalTok{ severity.x, }
\NormalTok{           mode) }\SpecialCharTok{\%\textgreater{}\%}
    
    \FunctionTok{mutate}\NormalTok{(}\AttributeTok{age\_range =} \FunctionTok{as.character}\NormalTok{(}\FunctionTok{cut}\NormalTok{(age,}
                                        \AttributeTok{breaks =} \FunctionTok{c}\NormalTok{(}\DecValTok{0}\NormalTok{, }\DecValTok{10}\NormalTok{, }\DecValTok{20}\NormalTok{, }\DecValTok{30}\NormalTok{, }\DecValTok{40}\NormalTok{, }\DecValTok{50}\NormalTok{,}
                                                   \DecValTok{60}\NormalTok{, }\DecValTok{70}\NormalTok{, }\DecValTok{80}\NormalTok{, }\DecValTok{90}\NormalTok{, }\DecValTok{100}\NormalTok{),}
                                        \AttributeTok{labels =} \FunctionTok{c}\NormalTok{(}\StringTok{"0{-}10"}\NormalTok{, }\StringTok{"11{-}20"}\NormalTok{,}
                                                   \StringTok{"21{-}30"}\NormalTok{, }\StringTok{"31{-}40"}\NormalTok{,}
                                                   \StringTok{"41{-}50"}\NormalTok{, }\StringTok{"51{-}60"}\NormalTok{,}
                                                   \StringTok{"61{-}70"}\NormalTok{, }\StringTok{"71{-}80"}\NormalTok{,}
                                                   \StringTok{"81{-}90"}\NormalTok{, }\StringTok{"91{-}100"}\NormalTok{))))}
  
  
  \FunctionTok{ggplot}\NormalTok{(plot\_df) }\SpecialCharTok{+}
    
    \FunctionTok{geom\_bar}\NormalTok{(}\FunctionTok{aes}\NormalTok{(}\AttributeTok{x =}\NormalTok{ age\_range, }
                 \AttributeTok{fill =}\NormalTok{ mode)) }\SpecialCharTok{+}
    
    \FunctionTok{guides}\NormalTok{(}\AttributeTok{fill =} \FunctionTok{guide\_legend}\NormalTok{(}\AttributeTok{title =} \StringTok{"Mode of Transport"}\NormalTok{)) }\SpecialCharTok{+}
    
    \FunctionTok{labs}\NormalTok{(}\AttributeTok{title =} \StringTok{"Casualties by Age"}\NormalTok{,}
         \AttributeTok{subtitle =} \StringTok{"21{-}30 year olds experience the highest number of transport accidents."}\NormalTok{,}
         \AttributeTok{x =} \StringTok{"Age"}\NormalTok{,}
         \AttributeTok{y =} \StringTok{"Accidents"}\NormalTok{) }\SpecialCharTok{+}
    
    \FunctionTok{theme\_minimal}\NormalTok{() }\SpecialCharTok{+}
    \FunctionTok{theme}\NormalTok{(}\AttributeTok{axis.text.x =} \FunctionTok{element\_text}\NormalTok{(}\AttributeTok{size =} \FunctionTok{rel}\NormalTok{(}\FloatTok{1.2}\NormalTok{)),}
          \AttributeTok{plot.title =} \FunctionTok{element\_text}\NormalTok{(}\AttributeTok{size=}\FunctionTok{rel}\NormalTok{(}\FloatTok{1.6}\NormalTok{), }\AttributeTok{color=}\StringTok{"\#5B2C6F"}\NormalTok{))}
\end{Highlighting}
\end{Shaded}

\includegraphics{publishreport_files/figure-latex/unnamed-chunk-18-1.pdf}

\hypertarget{age-range-for-borough-of-harrow}{%
\paragraph{Age Range for Borough of
Harrow}\label{age-range-for-borough-of-harrow}}

I wanted to take a look at the Borough of Harrow since it had the lowest
accidents per capita in all of London. We see that the accidents are
much more evenly spread out among different age groups.

\begin{Shaded}
\begin{Highlighting}[]
\NormalTok{plot\_df }\OtherTok{\textless{}{-}}\NormalTok{ accidents\_tbl }\SpecialCharTok{\%\textgreater{}\%} 
    \FunctionTok{inner\_join}\NormalTok{(casualties\_tbl, }\AttributeTok{by =} \FunctionTok{c}\NormalTok{(}\StringTok{"id"} \OtherTok{=} \StringTok{"accidentId"}\NormalTok{)) }\SpecialCharTok{\%\textgreater{}\%}
    \FunctionTok{select}\NormalTok{(borough, age, }\StringTok{"severity"} \OtherTok{=}\NormalTok{ severity.x, mode) }\SpecialCharTok{\%\textgreater{}\%}
    \FunctionTok{filter}\NormalTok{(borough }\SpecialCharTok{==} \StringTok{"Harrow"}\NormalTok{) }\SpecialCharTok{\%\textgreater{}\%}
    \FunctionTok{mutate}\NormalTok{(}\AttributeTok{age\_range =} \FunctionTok{as.character}\NormalTok{(}\FunctionTok{cut}\NormalTok{(age,}
                                        \AttributeTok{breaks =} \FunctionTok{c}\NormalTok{(}\DecValTok{0}\NormalTok{, }\DecValTok{10}\NormalTok{, }\DecValTok{20}\NormalTok{, }\DecValTok{30}\NormalTok{, }\DecValTok{40}\NormalTok{, }\DecValTok{50}\NormalTok{,}
                                                   \DecValTok{60}\NormalTok{, }\DecValTok{70}\NormalTok{, }\DecValTok{80}\NormalTok{, }\DecValTok{90}\NormalTok{, }\DecValTok{100}\NormalTok{),}
                                        \AttributeTok{labels =} \FunctionTok{c}\NormalTok{(}\StringTok{"0{-}10"}\NormalTok{, }\StringTok{"11{-}20"}\NormalTok{,}
                                                   \StringTok{"21{-}30"}\NormalTok{, }\StringTok{"31{-}40"}\NormalTok{,}
                                                   \StringTok{"41{-}50"}\NormalTok{, }\StringTok{"51{-}60"}\NormalTok{,}
                                                   \StringTok{"61{-}70"}\NormalTok{, }\StringTok{"71{-}80"}\NormalTok{,}
                                                   \StringTok{"81{-}90"}\NormalTok{, }\StringTok{"91{-}100"}\NormalTok{))))}
  
  
  \FunctionTok{ggplot}\NormalTok{(plot\_df) }\SpecialCharTok{+}
    
    \FunctionTok{geom\_bar}\NormalTok{(}\FunctionTok{aes}\NormalTok{(}\AttributeTok{x =}\NormalTok{ age\_range, }
                 \AttributeTok{fill =}\NormalTok{ mode)) }\SpecialCharTok{+}
    
    \FunctionTok{guides}\NormalTok{(}\AttributeTok{fill =} \FunctionTok{guide\_legend}\NormalTok{(}\AttributeTok{title =} \StringTok{"Mode of Transport"}\NormalTok{)) }\SpecialCharTok{+}
    
    \FunctionTok{labs}\NormalTok{(}\AttributeTok{title =} \StringTok{"Harrow Casualties by Age"}\NormalTok{,}
         \AttributeTok{subtitle =} \StringTok{"More evenly spread distribution throughout the age ranges, expected proportion of car{-}driver injuries"}\NormalTok{,}
         \AttributeTok{x =} \StringTok{"Age"}\NormalTok{,}
         \AttributeTok{y =} \StringTok{"Accidents"}\NormalTok{) }\SpecialCharTok{+}
    
    \FunctionTok{theme\_minimal}\NormalTok{() }\SpecialCharTok{+}
    \FunctionTok{theme}\NormalTok{(}\AttributeTok{axis.text.x =} \FunctionTok{element\_text}\NormalTok{(}\AttributeTok{size =} \FunctionTok{rel}\NormalTok{(}\FloatTok{1.2}\NormalTok{)),}
          \AttributeTok{plot.title =} \FunctionTok{element\_text}\NormalTok{(}\AttributeTok{size=}\FunctionTok{rel}\NormalTok{(}\FloatTok{1.6}\NormalTok{), }\AttributeTok{color=}\StringTok{"\#5B2C6F"}\NormalTok{))}
\end{Highlighting}
\end{Shaded}

\includegraphics{publishreport_files/figure-latex/unnamed-chunk-19-1.pdf}

\hypertarget{age-range-for-city-of-westminster}{%
\paragraph{Age range for City of
Westminster}\label{age-range-for-city-of-westminster}}

Severe or Fatal Cases

This chart continues to illustrate how relatively insignificant Car
injuries are when only focusing on Severe or Fatal injuries. We see the
majority of each age group occupied by Pedestrian or Bicycle injuries
and much less diversity overall in modes of transport.

\begin{Shaded}
\begin{Highlighting}[]
\NormalTok{plot\_df }\OtherTok{\textless{}{-}}\NormalTok{ accidents\_tbl }\SpecialCharTok{\%\textgreater{}\%} 
    \FunctionTok{inner\_join}\NormalTok{(casualties\_tbl, }\AttributeTok{by =} \FunctionTok{c}\NormalTok{(}\StringTok{"id"} \OtherTok{=} \StringTok{"accidentId"}\NormalTok{)) }\SpecialCharTok{\%\textgreater{}\%}
    \FunctionTok{select}\NormalTok{(borough, age, }\StringTok{"severity"} \OtherTok{=}\NormalTok{ severity.x, mode) }\SpecialCharTok{\%\textgreater{}\%}
    \FunctionTok{filter}\NormalTok{(borough }\SpecialCharTok{==} \StringTok{"City of Westminster"}\NormalTok{,}
\NormalTok{           severity }\SpecialCharTok{!=} \StringTok{"Slight"}\NormalTok{) }\SpecialCharTok{\%\textgreater{}\%}
    \FunctionTok{mutate}\NormalTok{(}\AttributeTok{age\_range =} \FunctionTok{as.character}\NormalTok{(}\FunctionTok{cut}\NormalTok{(age,}
                                        \AttributeTok{breaks =} \FunctionTok{c}\NormalTok{(}\DecValTok{0}\NormalTok{, }\DecValTok{10}\NormalTok{, }\DecValTok{20}\NormalTok{, }\DecValTok{30}\NormalTok{, }\DecValTok{40}\NormalTok{, }\DecValTok{50}\NormalTok{,}
                                                   \DecValTok{60}\NormalTok{, }\DecValTok{70}\NormalTok{, }\DecValTok{80}\NormalTok{, }\DecValTok{90}\NormalTok{, }\DecValTok{100}\NormalTok{),}
                                        \AttributeTok{labels =} \FunctionTok{c}\NormalTok{(}\StringTok{"0{-}10"}\NormalTok{, }\StringTok{"11{-}20"}\NormalTok{,}
                                                   \StringTok{"21{-}30"}\NormalTok{, }\StringTok{"31{-}40"}\NormalTok{,}
                                                   \StringTok{"41{-}50"}\NormalTok{, }\StringTok{"51{-}60"}\NormalTok{,}
                                                   \StringTok{"61{-}70"}\NormalTok{, }\StringTok{"71{-}80"}\NormalTok{,}
                                                   \StringTok{"81{-}90"}\NormalTok{, }\StringTok{"91{-}100"}\NormalTok{))))}
  
  \FunctionTok{ggplot}\NormalTok{(plot\_df) }\SpecialCharTok{+}
    
    \FunctionTok{geom\_bar}\NormalTok{(}\FunctionTok{aes}\NormalTok{(}\AttributeTok{x =}\NormalTok{ age\_range, }
                 \AttributeTok{fill =}\NormalTok{ mode)) }\SpecialCharTok{+}
    
    \FunctionTok{guides}\NormalTok{(}\AttributeTok{fill =} \FunctionTok{guide\_legend}\NormalTok{(}\AttributeTok{title =} \StringTok{"Mode of Transport"}\NormalTok{)) }\SpecialCharTok{+}
    
    \FunctionTok{labs}\NormalTok{(}\AttributeTok{title =} \StringTok{"City of Westminster Severe Casualties by Age"}\NormalTok{,}
         \AttributeTok{subtitle =} \StringTok{"Significantly more pedestrian injuries than car{-}driver injuries when compared to the London{-}wide statistics, especially for older travelers."}\NormalTok{,}
         \AttributeTok{x =} \StringTok{"Age"}\NormalTok{,}
         \AttributeTok{y =} \StringTok{"Accidents"}\NormalTok{) }\SpecialCharTok{+}
    
    \FunctionTok{theme\_minimal}\NormalTok{() }\SpecialCharTok{+}
    \FunctionTok{theme}\NormalTok{(}\AttributeTok{axis.text.x =} \FunctionTok{element\_text}\NormalTok{(}\AttributeTok{size =} \FunctionTok{rel}\NormalTok{(}\FloatTok{1.2}\NormalTok{)),}
          \AttributeTok{plot.title =} \FunctionTok{element\_text}\NormalTok{(}\AttributeTok{size=}\FunctionTok{rel}\NormalTok{(}\FloatTok{1.6}\NormalTok{), }\AttributeTok{color=}\StringTok{"\#5B2C6F"}\NormalTok{))}
\end{Highlighting}
\end{Shaded}

\includegraphics{publishreport_files/figure-latex/unnamed-chunk-20-1.pdf}

Only Car or Pedestrian Cases (Most Problematic)

When we pull out only Car and Pedestrian cases in City of Westminster
for all Severity types, we see that older travelers experience a higher
proportion of pedestrian injuries, but middle-aged travelers experience
more car injuries.

\begin{Shaded}
\begin{Highlighting}[]
\NormalTok{plot\_df }\OtherTok{\textless{}{-}}\NormalTok{ accidents\_tbl }\SpecialCharTok{\%\textgreater{}\%} 
    \FunctionTok{inner\_join}\NormalTok{(casualties\_tbl, }\AttributeTok{by =} \FunctionTok{c}\NormalTok{(}\StringTok{"id"} \OtherTok{=} \StringTok{"accidentId"}\NormalTok{)) }\SpecialCharTok{\%\textgreater{}\%}
    \FunctionTok{select}\NormalTok{(borough, age, }\StringTok{"severity"} \OtherTok{=}\NormalTok{ severity.x, mode) }\SpecialCharTok{\%\textgreater{}\%}
    \FunctionTok{filter}\NormalTok{(mode }\SpecialCharTok{==} \StringTok{"Pedestrian"} \SpecialCharTok{|}\NormalTok{ mode }\SpecialCharTok{==} \StringTok{"Car"}\NormalTok{) }\SpecialCharTok{\%\textgreater{}\%}
    \FunctionTok{mutate}\NormalTok{(}\AttributeTok{age\_range =} \FunctionTok{as.character}\NormalTok{(}\FunctionTok{cut}\NormalTok{(age,}
                                        \AttributeTok{breaks =} \FunctionTok{c}\NormalTok{(}\DecValTok{0}\NormalTok{, }\DecValTok{10}\NormalTok{, }\DecValTok{20}\NormalTok{, }\DecValTok{30}\NormalTok{, }\DecValTok{40}\NormalTok{, }\DecValTok{50}\NormalTok{,}
                                                   \DecValTok{60}\NormalTok{, }\DecValTok{70}\NormalTok{, }\DecValTok{80}\NormalTok{, }\DecValTok{90}\NormalTok{, }\DecValTok{100}\NormalTok{),}
                                        \AttributeTok{labels =} \FunctionTok{c}\NormalTok{(}\StringTok{"0{-}10"}\NormalTok{, }\StringTok{"11{-}20"}\NormalTok{,}
                                                   \StringTok{"21{-}30"}\NormalTok{, }\StringTok{"31{-}40"}\NormalTok{,}
                                                   \StringTok{"41{-}50"}\NormalTok{, }\StringTok{"51{-}60"}\NormalTok{,}
                                                   \StringTok{"61{-}70"}\NormalTok{, }\StringTok{"71{-}80"}\NormalTok{,}
                                                   \StringTok{"81{-}90"}\NormalTok{, }\StringTok{"91{-}100"}\NormalTok{))))}
  
  \FunctionTok{ggplot}\NormalTok{(plot\_df) }\SpecialCharTok{+}
    
    \FunctionTok{geom\_bar}\NormalTok{(}\FunctionTok{aes}\NormalTok{(}\AttributeTok{x =}\NormalTok{ age\_range, }
                 \AttributeTok{fill =}\NormalTok{ mode)) }\SpecialCharTok{+}
    
    \FunctionTok{scale\_fill\_manual}\NormalTok{(}\AttributeTok{values =} \FunctionTok{c}\NormalTok{(}\StringTok{"Car"} \OtherTok{=} \StringTok{"\#de8e00"}\NormalTok{,}
                                 \StringTok{"Pedestrian"} \OtherTok{=} \StringTok{"\#00bce8"}\NormalTok{)) }\SpecialCharTok{+}
    
    \FunctionTok{labs}\NormalTok{(}\AttributeTok{title =} \StringTok{"City of Westminster Car or Pedestrian Casualties by Age"}\NormalTok{,}
         \AttributeTok{subtitle =} \StringTok{"Older travelers experience a higher proportion of pedestrian injuries, but middle{-}aged travelers experience more driving injuries."}\NormalTok{,}
         \AttributeTok{x =} \StringTok{"Age"}\NormalTok{,}
         \AttributeTok{y =} \StringTok{"Accidents"}\NormalTok{) }\SpecialCharTok{+}
    
    \FunctionTok{theme\_minimal}\NormalTok{() }\SpecialCharTok{+}
    \FunctionTok{theme}\NormalTok{(}\AttributeTok{axis.text.x =} \FunctionTok{element\_text}\NormalTok{(}\AttributeTok{size =} \FunctionTok{rel}\NormalTok{(}\FloatTok{1.2}\NormalTok{)),}
          \AttributeTok{plot.title =} \FunctionTok{element\_text}\NormalTok{(}\AttributeTok{size=}\FunctionTok{rel}\NormalTok{(}\FloatTok{1.6}\NormalTok{), }\AttributeTok{color=}\StringTok{"\#5B2C6F"}\NormalTok{))}
\end{Highlighting}
\end{Shaded}

\includegraphics{publishreport_files/figure-latex/unnamed-chunk-21-1.pdf}

\hypertarget{future}{%
\section{⏭️ Future}\label{future}}

To continue with this project, I would be very interested in including
London Underground accidents and delays and how that relates to other
TfL accidents. I would gather info on tube stations per borough and
average business of each tube station (users per day or per month of
each station). This information would be far more valuable than the
population of each borough because the population doesn't necessarily
account for the number of people traveling, especially with the amount
of tourists that London reveives. This is likely one of the reasons why
I had to omit the City of London borough from my data, as it caused an
outlier for its high amount of accidents per capita.

In addition to London Underground data, I would like to dive deeper into
the issues of the City of Westminster. It is fascinating that it leads
both gross and per capita accidents, so I would like to continue
exploring what causes these accidents and who it affects. I would also
like to narrow down much tighter than boroughs. The TfL API included
location descriptions as well at latitude and longitude, so with enough
tidying, I could get much more specific on where the problem areas are
in Westminster. This would lead to more interesting and productive
conclusions.

Finally, I would like to span this information across several years to
see patterns in frequent modes of transport in accidents and where the
most dangerous areas have moved from. This allows us to lay timelines of
city infrastructure projects side by side with accidents plots and see
where infrastructure improvements have made differences in the
efficienty and safety of London transport. This is the overall
motivation of my interest in this topic because these networks heavily
impact our daily lives and are running constantly all around us.

\end{document}
